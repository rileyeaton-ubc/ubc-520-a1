This section analyzes the time and space complexity of five different data structures used for login checking: linear search, binary search, hash tables, Bloom filters, and Cuckoo filters.

\subsection{Linear Search}

Linear search stores elements in an unsorted array or list and searches sequentially through all elements.

\textbf{Parameters:}
\begin{itemize}
    \item $n$ = number of stored logins
\end{itemize}

\textbf{Time Complexity:}
\begin{itemize}
    \item Insert: $O(n)$ - must check all existing elements to verify uniqueness
    \item Search: $O(n)$ - worst case requires examining all elements
\end{itemize}

\textbf{Space Complexity:} $O(n)$ - stores exactly $n$ elements

Linear search provides no optimization for lookups, making it inefficient for large datasets \cite{cormen2009introduction}.

\subsection{Binary Search}

Binary search maintains elements in a sorted array, enabling logarithmic search time through the divide-and-conquer approach.

\textbf{Parameters:}
\begin{itemize}
    \item $n$ = number of stored logins
\end{itemize}

\textbf{Time Complexity:}
\begin{itemize}
    \item Insert: $O(n)$ - $O(\log n)$ for binary search to find position, but $O(n)$ for array shifting to maintain sorted order
    \item Search: $O(\log n)$ - divides search space in half at each step
\end{itemize}

\textbf{Space Complexity:} $O(n)$ - stores exactly $n$ elements in sorted order

Binary search significantly improves lookup performance but insertion remains costly due to the need to maintain sorted order \cite{cormen2009introduction}.

\subsection{Hash Tables}

Hash tables use a hash function to map keys to array indices, providing constant-time average case operations.

\textbf{Parameters:}
\begin{itemize}
    \item $n$ = number of stored logins
    \item $m$ = size of hash table
    \item $\alpha = n/m$ = load factor
\end{itemize}

\textbf{Time Complexity:}
\begin{itemize}
    \item Insert: $O(1)$ average case, $O(n)$ worst case with collisions
    \item Search: $O(1)$ average case, $O(n)$ worst case with collisions
\end{itemize}

\textbf{Space Complexity:} $O(n)$ - with good hash functions and proper load factor management

Hash tables provide excellent average-case performance when the load factor $\alpha$ is kept below a threshold (typically 0.7) \cite{cormen2009introduction}. Python's \texttt{set} implementation uses open addressing with a load factor that triggers resizing.

\subsection{Bloom Filters}

A Bloom filter is a probabilistic data structure that uses multiple hash functions to map elements into a bit array, allowing for space-efficient membership testing with possible false positives \cite{bloom1970space}.

\textbf{Parameters:}
\begin{itemize}
    \item $n$ = estimated number of elements to insert
    \item $m$ = size of bit array (in bits)
    \item $k$ = number of hash functions
    \item $p$ = desired false positive probability
\end{itemize}

\textbf{Time Complexity:}
\begin{itemize}
    \item Insert: $O(k)$ - requires $k$ hash function evaluations
    \item Search: $O(k)$ - requires $k$ hash function evaluations
\end{itemize}

\textbf{Space Complexity:} $O(m)$ bits

The optimal number of hash functions is given by:
\begin{equation}
k = \frac{m}{n} \ln 2
\end{equation}

The optimal bit array size for a desired false positive probability $p$ is:
\begin{equation}
m = -\frac{n \ln p}{(\ln 2)^2}
\end{equation}

The false positive probability after inserting $n$ elements is approximately:
\begin{equation}
p \approx \left(1 - e^{-kn/m}\right)^k
\end{equation}

Bloom filters trade accuracy for space efficiency, making them ideal when false positives are acceptable but false negatives are not \cite{broder2004network}. My implementation uses a backing hash table to verify positive results and eliminate false positives.

\subsection{Cuckoo Filters}

Cuckoo filters extend Bloom filters by storing fingerprints of items using cuckoo hashing, enabling deletions while maintaining space efficiency \cite{fan2014cuckoo}.

\textbf{Parameters:}
\begin{itemize}
    \item $n$ = number of elements to insert
    \item $m$ = number of buckets
    \item $b$ = bucket size (entries per bucket)
    \item $f$ = fingerprint size (in bits)
    \item $\alpha$ = load factor (typically $\leq 0.95$)
\end{itemize}

\textbf{Time Complexity:}
\begin{itemize}
    \item Insert: $O(1)$ average case, with a small probability of failure requiring rehashing
    \item Search: $O(1)$ - checks at most 2 buckets with $b$ entries each
    \item Delete: $O(1)$ - unlike Bloom filters, supports deletion
\end{itemize}

\textbf{Space Complexity:} $O(n \cdot f)$ bits, where $f$ is typically 4-16 bits

The false positive rate is approximately:
\begin{equation}
\epsilon \approx \frac{2b}{2^f}
\end{equation}

where $f$ is the fingerprint size in bits and $b$ is the bucket size. For a load factor $\alpha$, the total capacity is $C = \alpha \cdot b \cdot m$ \cite{fan2014cuckoo}.

Cuckoo filters provide similar space efficiency to Bloom filters while supporting deletion and offering better lookup performance for certain parameters.

\subsection{(Bonus) Tries}

A trie (prefix tree) is a tree-based data structure where each node represents a character and paths from root to leaf represent complete strings, making it well-suited for string operations \cite{cormen2009introduction}.

\textbf{Parameters:}
\begin{itemize}
    \item $n$ = number of stored logins
    \item $m$ = average length of login strings
    \item $\sigma$ = alphabet size (typically $\approx$ 100 for alphanumeric + symbols)
\end{itemize}

\textbf{Time Complexity:}
\begin{itemize}
    \item Insert: $O(m)$ - traverse/create $m$ nodes
    \item Search: $O(m)$ - traverse path of length $m$
    \item Delete: $O(m)$ - locate and remove the string
\end{itemize}

\textbf{Space Complexity:} $O(n \cdot m \cdot \sigma)$ worst case, but much better with shared prefixes

Tries offer deterministic performance without hash collisions and support efficient prefix operations ($O(m + k)$ to find all logins with a given prefix). However, space overhead is higher than hash tables when strings lack common prefixes, and $O(m)$ time is slower than hash table's $O(1)$ for typical login lengths.

\subsection{Complexity Comparison}

Table~\ref{tab:complexity} summarizes the time and space complexity of all five approaches.

\begin{table}[h]
\centering
\caption{Computational Complexity Comparison}
\label{tab:complexity}
\begin{tabular}{@{}lccc@{}}
\toprule
\textbf{Data Structure} & \textbf{Insert} & \textbf{Search} & \textbf{Space} \\
\midrule
Linear Search & $O(n)$ & $O(n)$ & $O(n)$ \\
Binary Search & $O(n)$ & $O(\log n)$ & $O(n)$ \\
Hash Table & $O(1)^*$ & $O(1)^*$ & $O(n)$ \\
Bloom Filter & $O(k)$ & $O(k)$ & $O(m)$ bits \\
Cuckoo Filter & $O(1)^*$ & $O(1)$ & $O(nf)$ bits \\
\bottomrule
\end{tabular}
\begin{tablenotes}
\small
\item $^*$Average case complexity; worst case is $O(n)$
\item $n$ = number of elements, $k$ = number of hash functions
\item $m$ = bit array size, $f$ = fingerprint size in bits
\end{tablenotes}
\end{table}

